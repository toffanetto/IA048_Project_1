%%%%%%%%%%%%%%%%%%%%%%%%%%%%%%%%%%%%%%%%%%%%%%%%%%%
%%%%%%%%%%%%%%%%%%% toffanetto %%%%%%%%%%%%%%%%%%%%
% Template para atividades acadêmicas diversas    %
% Init: 2020 -- Update: 2021                      %
%%%%%%%%%%%%%%%%%%%%%%%%%%%%%%%%%%%%%%%%%%%%%%%%%%%

\documentclass{atividade_fftt}

%----Dados sobre a atividade----%
\def\Faculdade{Faculdade de Engenharia Elétrica e de Computação}
\def\Disciplina{Aprendizado de Máquina}   
\def\DisciplinaCOD{IA048}                                     
\def\Professor{Levy Boccato \& Romis Attux}                                      
\def\Titulo{Proposta de Tema -- Projeto Final}                                          
\def\TituloShort{}                                           
\def\Autor{Gabriel Toffanetto França da Rocha -- 289320} 
%-------------------------------%

%%%%%%%%%%%%%%%%%
% Configurações %
%%%%%%%%%%%%%%%%%

%% Text Font %%
%\timesnewroman 
%\arial
%\latinmodern

%% Using -- in itemize %%
\tracolista

%% PDF metadata setup %%
\pdfconfig


%---------------

\begin{document}

%%%%%%%%%%%%%%%%%
%   Estrutura   %
%%%%%%%%%%%%%%%%%

%\cabecalho
%\cabecalholite
%% Use alone for single author document %%
\cabecalhoalone
%\cabecalholitealone

%% Author name | \Autor by default %%
\autor
%\autor[Aluno -- RA]

%% Document title %%
\titulo
%\titulolite

%% Insert List of Contents %%
%\sumario

%% Using one and halt spacing between lines %%
\espacamentoumemeio

%% Remove header and footer
%\removeheader

%---------------

%%%%%%%%%%%%%%% Conteúdo da atividade %%%%%%%%%%%%%%%

%% Include corpo.tex file with document body %%
%\corpo

\section*{Segmentação Semântica de Imagens para Percepção de \\Veículos Autônomos}

Para que um robô possa se locomover pelo ambiente de forma autônoma, é preciso que ele tenha a capacidade de perceber o local onde está inserido, para poder interagir com ele de forma coerente. No contexto da navegação urbana de veículos autônomos, o veículo deve ser capaz de identificar no ambiente o que é rua, calçada, pedestres, postes, árvores, outros veículos, etc., com principal objetivo de diferir o que é área navegável do que são obstáculos. Existem muitos métodos e sensores que podem ser utilizados para tal tarefa, como por exemplo LiDARs e radares. Contudo, de forma mais natural, a câmera é o sensor que mais se aproxima da maneira como nós, humanos, percebemos o mundo. Assim, a utilização de imagens leva ao problema de definir, na matriz de pixels que as formam, a qual classe cada um deles pertence, realizando assim uma segmentação semântica da imagem, onde, além de detectar quais entidades estão presentes na cena, define-se com precisão onde elas estão e o espaço que ocupam no campo visual.

Dessa forma, propõe-se a utilização de algoritmos de Machine Learning, baseados em redes neurais profundas, para realizar, por meio de aprendizado supervisionado, a segmentação semântica de imagens de cenas urbanas, com foco em problemas de navegação de veículos autônomos. Dentre as tecnologias a serem utilizadas, citam-se as \textit{Fully Convolutional Networks} (FCNs) e os \textit{Autoencoders}, além de outras propostas de sucesso da literatura para tal aplicação. Além disso, devido à atividade custosa de criar rótulos para esse tipo de problema, pretende-se analisar a relevância do uso de \textit{Self-Supervised Learning} (SSL) para o pré-treino da rede neural profunda, sendo assim demandando menos amostras rotuladas do problema-alvo.



%----------------------------------------------------
% Citação extra, sem aparecer no texto

%\nocite{autor}

%----------------------------------------------------

%% Add bibliography %%
%\bibliografia
%\bibliografiaintext


%% Add attachment %%
%\anexos

\end{document}
%----------------------------------------------------
%                   Links uteis
% http://www.tablesgenerator.com/                                              % tabela em latex
% http://truben.no/latex/bibtex/                                               % referencias
% https://www.overleaf.com/learn/latex/Inserting_Images                        % imagens
% https://www.overleaf.com/learn/latex/positioning_images_and_tables           % posicionamento
% https://www.overleaf.com/learn/latex/List_of_Greek_letters_and_math_symbols  % simbolos gregos
% https://docx2latex.com/tutorials/MaTrix-LaTeX.html                           % matrizes
% https://formulasheet.com/                                                    % matrizes
% https://kogler.wordpress.com/2008/03/21/latex-multiline-equations-systems-and-matrices/
%----------------------------------------------------