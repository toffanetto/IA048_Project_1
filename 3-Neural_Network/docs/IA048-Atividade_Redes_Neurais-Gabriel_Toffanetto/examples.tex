%imagem

\begin{figure}[H]
    \centering
    \includegraphics[width = \textwidth]{img/familia_pid.jpg}
    \caption{Comparação das respostas ao degrau das diferentes variações do compensador PID.}
    \label{fig:familia}
    \begin{flushright}
        Confeccionado pelos autores (2021).
    \end{flushright}
\end{figure} 


\begin{figure}[H]
    \centering
    \includegraphics[width = \textwidth]{img/familia_pid.jpg}
    \caption{Comparação das respostas ao degrau das diferentes variações do compensador PID.}
    \label{fig:familia}
\end{figure} 

% codigo fonte
\begin{lstlisting}[language=Octave]
std::cout << "Hello wordl" << std::endl;
\end{lstlisting}

\lstinputlisting[language=Octave]{img/atividade3.m}


%sistema linear
\begin{equation*}
     \begin{cases} 
        x+2y+z = 8\\ 
        2x-y+z = 3\\ 
        3x+y-z = 2 
    \end{cases}   
\end{equation*}


%%% MULTI COLUNAS %%%%

\begin{multicols}{2}[]

%text

\end{multicols}

%%% SUBFIGURE %%%

\begin{figure}[H]
    \centering
    \begin{subfigure}[H]{0.4\textwidth}
        \includegraphics[width = \textwidth]{img/circuit1.png}
        \caption{Circuito.}
        \label{fig:circuito1}
    \end{subfigure}
     \begin{subfigure}[H]{0.5\textwidth}
        \centering
        \includegraphics[width = \textwidth]{img/pequenos_sinais.jpg}
        \caption{Modelo de pequenos sinais para o circuito.}
        \label{fig:pequenossinais}
     \end{subfigure}
\end{figure}

%%% VECTOR IMAGE %%%

\begin{figure}[H]
        \centering
        \footnotesize
        \includesvg[width = 0.9\textwidth]{img/GQ1.svg}
        \normalsize
        \caption{ Gráfico que representa o comportamento de ft x VGS}
        \label{fig:steptemp}
\end{figure}